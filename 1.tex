\documentclass[tikz,border=10pt]{standalone}
\usepackage{tikz}
\usetikzlibrary{shapes.geometric, arrows.meta, positioning, fit, backgrounds, calc}

% 定义基本样式
\tikzset{
    base/.style = {draw, text centered, minimum height=3em, font=\sffamily},
    process/.style = {base, rectangle, rounded corners, fill=blue!10, minimum width=4cm, align=center},
    io/.style = {base, trapezium, trapezium left angle=70, trapezium right angle=110, fill=orange!10, minimum width=3cm, align=center},
    cloud/.style = {base, ellipse, fill=green!10, minimum width=4cm, minimum height=2em},
    layerbox/.style = {draw, dashed, inner sep=1em, rounded corners, fill=gray!5},
    arrow/.style = {thick, ->, >=Stealth}
}

\begin{document}

\begin{tikzpicture}[node distance=1.5cm and 2cm]

    % --- 1. User Layer Nodes ---
    \node (userstart) [io, fill=red!10] {User generates Raw Trajectory \\ \& Needs LBS};
    \node (calp) [process, below=of userstart] {\textbf{Step 1: User-Side Privacy}\\ Context-Aware Location Perturbation \\ (CALP) Mechanism\\ \textit{[Technique: Differential Privacy (Exponential Mechanism)]}};
    \node (perturbedData) [io, below=of calp, font=\footnotesize] {Perturbed Location/POI Data};

    % --- 2. Edge Layer Nodes ---
    \node (edgeProcess) [process, below=of perturbedData] {\textbf{Step 2: Edge Processing}\\ Edge-based Semantic Trajectory \\ Reconstruction (ESTR) Mechanism\\ \textit{[Goal: Compress data, identify Stay/Move points]}};
    \node (semanticData) [io, below=of edgeProcess, font=\footnotesize] {Semantic Trajectory Representation};

    % --- 3. Cloud Layer Nodes ---
    \node (cloudProcess) [process, below=of semanticData] {\textbf{Step 3: Cloud Optimization}\\ Cloud-based Trajectory Learning \\ (CBTL) Mechanism\\ \textit{[Technique: Bi-LSTM Deep Learning Model]}};
    \node (finalResult) [cloud, below=of cloudProcess, fill=green!20] {Optimized & Predicted \\ Complete Trajectory};

    % --- Arrows representing Data Flow ---
    \draw [arrow] (userstart) -- node[right, font=\footnotesize] {Raw Location} (calp);
    \draw [arrow] (calp) -- (perturbedData);
    \draw [arrow] (perturbedData) -- node[right, font=\footnotesize] {Secure Transmission} (edgeProcess);
    \draw [arrow] (edgeProcess) -- (semanticData);
    \draw [arrow] (semanticData) -- node[right, font=\footnotesize] {Low-Bandwidth Transmission} (cloudProcess);
    \draw [arrow] (cloudProcess) -- (finalResult);

    % Feedback loop arrow back to user
    \draw [arrow, draw=green!50!black, very thick, rounded corners] (finalResult.east) -- ++(2.5,0) |- node[pos=0.25, right, align=center, font=\sffamily\small] {High-Quality LBS Result \\ Returned to User} ($(userstart.east)!0.5!(calp.east)$) -- ($(userstart.east)!0.5!(calp.east) - (1,0)$);


    % --- Layer Background Boxes and Labels ---
    \begin{pgfonlayer}{background}
        % User Layer Box
        \node (userlayer) [layerbox, fit=(userstart) (calp) (perturbedData), fill=red!5, label={[anchor=north west, font=\bfseries\sffamily, text=red!70!black]north west:User Layer (Device)}] {};

        % Edge Layer Box
        \node (edgelayer) [layerbox, fit=(edgeProcess) (semanticData), below=of userlayer, yshift=0.5cm, fill=blue!5, label={[anchor=north west, font=\bfseries\sffamily, text=blue!70!black]north west:Edge Layer (Base Station/AP)}] {};

        % Cloud Layer Box
        \node (cloudlayer) [layerbox, fit=(cloudProcess) (finalResult), below=of edgelayer, yshift=0.5cm, fill=green!5, label={[anchor=north west, font=\bfseries\sffamily, text=green!70!black]north west:Cloud Layer (Data Center)}] {};
    \end{pgfonlayer}

\end{tikzpicture}

\end{document}